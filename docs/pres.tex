\documentclass[9pt]{beamer}
\usepackage[T1,T2A]{fontenc}
\usepackage[utf8]{inputenc}
\usepackage{hyperref}
\hypersetup{unicode=true}
\usepackage{fontawesome}
\usepackage{graphicx}
\usepackage{bookmark}
\usepackage[english,russian]{babel}

\usetheme{metropolis}
\usecolortheme{beaver}

\title{Разработка приложения <<Тир>>}
\subtitle{Отчет о проектной работе по курсу <<Основы информатики и программирования>>}
\author{Мельников Илья Евгеньевич и Михеева Валерия Александровна}
\date{11 июня 2021}

\begin{document}

\maketitle

\begin{frame}[fragile]{Цель проекта}
    Цель работы -- разработать приложение «Тир» на С++ и QML (QtQuick).
\end{frame}

\begin{frame}
    \frametitle{Этапы разработки приложения}
    \begin{itemize}
        \item Разработать графический интерфейс на QML
        \item Реализовать классы C++ для описания объектов игры
        \item Реализовать обработку клика-"выстрела"
        \item Реализовать систему начисления очков
        \item Реализовать логику игры
    \end{itemize}
\end{frame}

\begin{frame}[fragile]{Результат}
    В результате была разработана игра "Тир". При разработке приложения учитывались все требования, которыми должна обладать программа, разработанная для использования начинающими пользователями ПЭВМ, поэтому интерфейс игры довольно простой и понятный.
    \begin{itemize}
        \item Управление игрой с клавиатуры
        \item Физика шарика
        \item Обработка столкновений
        \item Обработка завершения игры (победа или поражение)
    \end{itemize}

    \begin{center}
        \includegraphics[scale=0.3]{iyI0b74Ajus.jpg}
    \end{center}

\end{frame}

\end{document}
